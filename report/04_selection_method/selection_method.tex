\section{Selection Method}
To generate a new population, specific genetic operators are applied to members of an existing population with specific probabilities. These operators are explained in more detail in section \ref{sec:genetic_operators}. However, before these operators can be applied, the members of the existing population to which they need to be applied must be determined - hence, a selection method is necessary.

For this assignment, tournament selection \cite{miller1995genetic} was used as the selection method for all genetic operators. This entails selecting \emph{T} individuals entirely at random from the existing population. Then, from this set of individuals (ie. the \emph{tournament}) the fittest individual is selected. After some initial test runs the size, \emph{T}, of the tournament used was 48. On average, this tournament size produced the best balance between genetic diversity and convergence.

In the cases where genetic operators required more than one individual from the existing population, the tournament selection method was simply reapplied to select each individual.