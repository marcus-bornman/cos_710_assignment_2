\section{Representation}

\subsection{Nodes}
The possible nodes for each individual, or decision tree, comprise decision nodes and end nodes. Decision nodes take a certain number of children and select one of the children based on the value of one of the patient's attributes. End nodes represent one of the three available discharge decisions.

There are 7 decision nodes - one for each available patient attribute. The number of children for each decision node is dependent on the number of possible values for the associated attribute. The Internal Temperature (\emph{IT}) node, for instance, accepts 3 children - the first is selected when the patient's internal temperature is \emph{low}, the second when it is \emph{mid} and the third when it is \emph{high}.

In addition, there are 3 end nodes - \emph{I}, \emph{S} and \emph{A}. These end nodes, as well as the aforementioned decision nodes, are described in more detail in table \ref{tab:nodes}.

\begin{table}[H]
\resizebox{\textwidth}{!}{\begin{tabular}{|l|l|l|}
\hline
\textbf{Symbol} & \textbf{Arity} & \textbf{Description}                                                                                                                                                                                                                                                                                                                                                                           \\ \hline
IT              & 3              & \begin{tabular}[c]{@{}l@{}}The internal temperature decision node\\ - If the patient's internal temperature is low then select the first child\\ - If the patient's internal temperature is mid then select the second child\\ - If the patient's internal temperature is high then select the third child\end{tabular}                                                                        \\ \hline
ITS             & 3              & \begin{tabular}[c]{@{}l@{}}The internal temperature stability decision node\\ - If the patient's internal temperature is unstable then select the first child\\ - If the patient's internal temperature is moderately stable then select the second child\\ - If the patient's internal temperature is stable then select the third child\end{tabular}                                         \\ \hline
ST              & 3              & \begin{tabular}[c]{@{}l@{}}The surface temperature decision node\\ - If the patient's surface temperature is low then select the first child\\ - If the patient's surface temperature is mid then select the second child\\ - If the patient's surface temperature is high then select the third child\end{tabular}                                                                            \\ \hline
STS             & 3              & \begin{tabular}[c]{@{}l@{}}The surface temperature stability decision node\\ - If the patient's surface temperature is unstable then select the first child\\ - If the patient's surface temperature is moderately stable then select the second child\\ - If the patient's surface temperature is stable then select the third child\end{tabular}                                             \\ \hline
BP              & 3              & \begin{tabular}[c]{@{}l@{}}The blood pressure decision node\\ - If the patient's blood pressure is low then select the first child\\ - If the patient's blood pressure is mid then select the second child\\ - If the patient's blood pressure is high then select the third child\end{tabular}                                                                                                \\ \hline
BPS             & 3              & \begin{tabular}[c]{@{}l@{}}The blood pressure stability decision node\\ - If the patient's blood pressure is unstable then select the first child\\ - If the patient's blood pressure is moderately stable then select the second child\\ - If the patient's blood pressure is stable then select the third child\end{tabular}                                                                 \\ \hline
OS              & 4              & \begin{tabular}[c]{@{}l@{}}The oxygen saturation decision node\\ - If the patient's oxygen saturation is poor then select the first child\\ - If the patient's oxygen saturation is fair then select the second child\\ - If the patient's oxygen saturation is good then select the third child\\ - If the patient's oxygen saturation is excellent then select the fourth child\end{tabular} \\ \hline
I               & 0              & The end node for the patient to be sent to an intensive care unit                                                                                                                                                                                                                                                                                                                              \\ \hline
S               & 0              & The end node for the patient to be sent home                                                                                                                                                                                                                                                                                                                                                   \\ \hline
A               & 0              & The end node for the patient to be sent to the general hospital floor                                                                                                                                                                                                                                                                                                                          \\ \hline
\end{tabular}}
\caption{Decision Tree Nodes}
\label{tab:nodes}
\end{table}

\subsection{Individuals}\label{sec:individuals}
Each chromosome within a population consists of a single decision tree for determining the discharge decision for a patient. To clarify, internal nodes indicate the attributes of a patient while links represent each of the possible values for an attribute. Given the possible nodes identified in table \ref{tab:nodes}, an example of one such decision tree is depicted in figure \ref{fig:decision_tree}.

% minimum distance between nodes on the same line
\setlength{\GapWidth}{1em}  
% draw with a thick dashed line, very nice looking
\thicklines \drawwith{\dottedline{2}}   
% draw an oval and center it with the rule.  You may want to fool with the
% rule values, though these seem to work quite well for me.  If you make the
% rule smaller than the text height, then the GP nodes may not line up with
% each other horizontally quite right, so watch out.
\newcommand{\gpbox}[1]{\Ovalbox{#1\rule[-.7ex]{0ex}{2.7ex}}}

\begin{figure}[H]
    \centering
    \resizebox{\textwidth}{!}{\begin{bundle}{\gpbox{ITS}}\chunk{\begin{bundle}{\gpbox{BPS}}\chunk{\begin{bundle}{\gpbox{BP}}\chunk{\gpbox{A}}\chunk{\gpbox{A}}\chunk{\gpbox{A}}\end{bundle}}\chunk{\begin{bundle}{\gpbox{OS}}\chunk{\gpbox{A}}\chunk{\gpbox{A}}\chunk{\gpbox{A}}\chunk{\gpbox{S}}\end{bundle}}\chunk{\begin{bundle}{\gpbox{BPS}}\chunk{\gpbox{I}}\chunk{\gpbox{A}}\chunk{\gpbox{S}}\end{bundle}}\end{bundle}}\chunk{\begin{bundle}{\gpbox{ST}}\chunk{\begin{bundle}{\gpbox{IT}}\chunk{\gpbox{A}}\chunk{\gpbox{I}}\chunk{\gpbox{I}}\end{bundle}}\chunk{\begin{bundle}{\gpbox{STS}}\chunk{\gpbox{I}}\chunk{\gpbox{I}}\chunk{\gpbox{I}}\end{bundle}}\chunk{\begin{bundle}{\gpbox{ITS}}\chunk{\gpbox{S}}\chunk{\gpbox{A}}\chunk{\gpbox{S}}\end{bundle}}\end{bundle}}\chunk{\begin{bundle}{\gpbox{IT}}\chunk{\begin{bundle}{\gpbox{ST}}\chunk{\gpbox{A}}\chunk{\gpbox{A}}\chunk{\gpbox{A}}\end{bundle}}\chunk{\begin{bundle}{\gpbox{ITS}}\chunk{\gpbox{S}}\chunk{\gpbox{I}}\chunk{\gpbox{A}}\end{bundle}}\chunk{\begin{bundle}{\gpbox{ITS}}\chunk{\gpbox{I}}\chunk{\gpbox{S}}\chunk{\gpbox{A}}\end{bundle}}\end{bundle}}\end{bundle}}
    \caption{Decision Tree}
    \label{fig:decision_tree}
\end{figure}

\subsection{Populations}
Of course, a population consists of a collection of the individuals that have been described in section \ref{sec:individuals}. Importantly, the genetic algorithm applied only made use of a single population consisting of 2048 individuals. This means that only a single population was evolved and cross-population breeding was not applied.

The initial population was generated using the ramped-half-and-half method proposed by Koza \cite{koza1992genetic} to enhance population diversity of structure. The maximum tree depth allowed during the initial population generation was 4. So, in other words, the population is divided equally among individuals to be initialized with trees having depths 2, 3, and 4. For each depth group, half of the trees are initialized with the full technique and half with the grow technique.

In addition, to cover as much of the search space as possible an effort was made to reduce duplicates within the initial population. The strategy used to reduce the number of duplicates was as follows: if a duplicate individual is created as part of the initial population, it will be regenerated up to 100 times to try replace it with a new, original individual. After 100 attempts, if no unique individual was generated, the duplicate individual will be used.